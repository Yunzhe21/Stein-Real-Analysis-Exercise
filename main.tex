\documentclass[12pt]{article}
\usepackage[margin=1in]{geometry}
\usepackage[all]{xy}


\usepackage{amsmath,amsthm,amssymb,color,latexsym}
\usepackage{geometry}        
\geometry{letterpaper}    
\usepackage{graphicx}

\newtheorem{problem}{Problem}

\newenvironment{solution}[1][\it{Solution}]{\textbf{#1. } }{$\square$}


\begin{document}
\noindent Stein Real Analysis \hfill Chapter 3\\
Yunzhe Zheng. (2025/01)

\hrulefill

\begin{problem}
Suppose $\varphi$ is an integrable function on $\mathbb{R}^{d}$ with $\int_{\mathbb{R}^{d}}\varphi(x)dx = 1$. Set $K_{\delta}(x) = \delta^{-d}\varphi(x/\delta)$, $\delta > 0$. \\

(1). Prove that $\{K_{delta}\}_{\delta>0}$ is a family of good kernels. \\
\indent (2). Assume in addition that $\varphi$ is bounded and supported in a bounded set. Verify that $\{K_{\delta}\}_{\delta>0}$ is an approximation to the identity. \\
\indent (3). Show that Theorem 2.3 holds for good kernels as well.
\end{problem}

\textbf{Proof:} (1). Firstly, $\int_{\mathbb{R}^{d}}K_{\delta}(x)dx = \int_{\mathbb{R}^{d}}\delta^{-d}\varphi(x/\delta)dx$, and by change of variable, it's identical to $\int_{\mathbb{R}^{d}}\varphi(u)du = 1$. Secondly, $\int_{\mathbb{R}^{d}}\left | \delta^{-d} \varphi(x/\delta)\right | = \delta^{-d}\int_{\mathbb{R}^{d}}|\varphi(x/\delta)|dx$, and by change of variable and the fact that $\varphi$ is lebesgue integrable, $\int_{\mathbb{R}^{d}}|\delta^{-d}\varphi(x/\delta)|dx \leq A$ for some $A$ independent of $\delta$. Finally, for fixed $\eta$,  by change of variable, $\int_{\mathbb{R}^{d}}\delta^{-d}|\varphi(x/\delta)|dx = \int_{u>\eta / \delta}|\varphi(u)|du$. Since $\eta / \delta\to\infty$ as $\delta\to 0$, thus the integral approaches 0 by integrability. \\

(2). Suppose that for $|x| > B$, $\varphi(x) = 0$, and $\sup\varphi(x) = M$. Setting $A = MB^{d+1}$, then $\delta^{-d}\varphi(x/\delta)< \frac{M|x/\delta|^{d+1}}{|x|^{d+1}} < \frac{A}{|x|^{d+1}}$ for $\frac{|x|}{\delta}\leq B$, otherwise it's zero. If we pick $B$ large enough ($B > 1$), then $|K_{\delta}(x)| \leq MB^{d+1}\delta^{-d} = A\delta^{-d}$ is obvious. Thus $K_{\delta}(x)$ is an approximation to identity. \\

(3). To prove integrability, we utilize Fubini Theorem, and the convolution is integrable. Secondly, compute

\begin{eqnarray*}
\int_{\mathbb{R}^{d}}(\int_{\mathbb{R}^{d}}f(x-y)K_{\delta}(y)dy - f(x))dx 
&=& \int_{R^{d}}\int_{R^{d}}\left [ f(x-y) - f(x)\right]K_{\delta}(y)dydx \\
&=& \int_{\mathbb{R}^{d}}K_{\delta}(y)\int_{\mathbb{R^{d}}}(f(x-y) - f(x))dxdy \\
&=& \int_{|y| > \eta}K_{\delta}(y)\int_{\mathbb{R}^{d}}(f(x-y) - f(y))dxdy \\
&+& \int_{|y|\leq \eta}K_{\delta}(y)\int_{\mathbb{R}^{d}}(f(x-y)  - f(y))dxdy 
\end{eqnarray*}, where $\eta$ is chosen so that $\int_{\mathbb{R}^{d}}(f(x - y) - f(x))<\epsilon$, thus $\|(f \ast K_{\delta}) - f\|_{1}\to 0$ as $\delta\to 0$. \qed


\begin{problem}
Suppose that $\{K_{\delta}\}$ is a family of kernels that satisfies: \\

(i). $|K_{\delta}(x)|\leq A\delta^{-d}$ for all $\delta > 0$. \\
\indent (ii). $|k_{\delta}(x)|\leq A\delta/|x|^{d+1}|$ for all $\delta > 0$. \\ 
\indent (iii). $\int_{-\infty}^{\infty}K_{\delta}(x)dx = 0$ for all $\delta > 0$. \\

Show that if $f$ is integrable on $\mathbb{R}^{d}$, then $(f\ast K_{\delta})(x) \to 0$ for almost every $x$ as $\delta\to 0$.
\end{problem}

\textbf{Proof:} Notice that $\left | (f \ast K_{\delta})(x) \right | = \left | \int_{\mathbb{R^{d}}} f(x-y)K_{\delta}(y)dy \right | = \left | \int_{\mathbb{R}^{d}}f(x-y)K_{\delta}(y) - \int_{\mathbb{R}^{d}}f(x)K_{\delta}(y)\right |$. Then the proof follows exactly the same as in Theorem 2.1 Chapter 3. \qed

\begin{problem}
Suppose $0$ is a point of Lebesgue density of the set $E\subset\mathbb{R}$. Show that for each of the individual conditions below there is an infinite sequence of points $\{x_{n}\}\subset E$, with $x_{n}\neq 0$, and $x_{n}\to 0$ as $n\to \infty$. \\

(a). The sequence also satisfies $-x_{n}\in E$ for all $n$. \\
\indent (b). In addition, $2x_{n}$ belongs to $E$ for all $n$.
\end{problem}

\textbf{Proof:} (a). We confine ourselves in the ball $B_{1}$ centered at $0$ with a radius of $1$. Choose $\{x_{n}\}$ in the following way: $x_{1}$ is the element in $(0, 1)\cap E$ such that $-x_{n}\in B_{1}\cap E$, for if such $x_{1}$ doesn't exists, then for every $x\in (0, 1)\cap E$, $-x\notin (-1, 0)\cap E$, thus we can find a sequence of sets where $S_{i}$ is the Ball centered at $0$ with a radius of $1/i$, and $\frac{m(B_{i}\cap E)}{m(B)}\leq \frac{1}{2}$, then $0$ cannot be a point of Lebesgue density. $x_{i}$ can be defined recursively in the ball $B_{0}(x_{i - 1})$. Thus we construct a sequence as desired. \\

(b). The idea is to show that $E\cap \frac{1}{2}E$ has positive measure, then since $0\in (\frac{1}{2}E)\cap E$, there exists a sequence $\{x_{n}\}$ in $\frac{1}{2}E$ (in fact in $\frac{1}{2}E)\cap E$), such that $2x_{n}\in E$. \\ 
\indent To prove that, since $0$ is a point of Lebesgue density, then there exists $r_{0}$ such that for $r \leq r_{0}$, $m(E\cap B_{r}(0)) > \frac{2}{3}m(B_{r}(0)) = \frac{4}{3}r$, then $m(\frac{1}{2}E\cap B_{r_{0}/2}(0))>\frac{2}{3}r$. Also, we have $m(E\cap B_{r_{0}/2}(0)) >\frac{2}{3}m(B_{r_{0}/2}(0)) = \frac{2}{3}r$, then since $2\times \frac{2}{3} > 1 = m(B_{r_{0}/2}(0))$, $E$ and $\frac{1}{2}E$ intersects and have positive measure. \qed

\begin{problem}
Prove that if $f$ is integrable on $\mathbb{R}^{d}$, and $f$ is not identically zero, then 
$$
f^{*}(x)\geq \frac{c}{|x|^{d}}, \text{ for some } c > 0 \text{ and all } |x| \geq 1.
$$ Conclude that $f^{*}$ is not integrable on $\mathbb{R}^{d}$. Then show that the weak type estimate $m(\{x: f^{*}(x) > \alpha\}) \leq c/\alpha$ for all $\alpha > 0 $ whenever $\int |f| = 1$, is the best possible in the following sense: if $f$ is supported in the unit ball with $\int |f| = 1$, then   
$$
m(\{x : f^{*}(x) > \alpha\}) \geq c'/\alpha
$$ for some $c' > 0$ and all sufficiently small $\alpha$.
\end{problem}

\textbf{Proof:} Since $f$ is not identically zero, there exists $B_{r}(0)$ such that $\int_{B_{r}(0)}|f(x)|dx > 0$. For $x$, construct a ball centered at it with a radius of $R = |x| + r$, containing $B_{r}(0)$. Then, 
$$
f^{*}(x) > \frac{1}{m(B_{R}(x))}\int_{B_{R}(x)}|f(y)|dy = \frac{1}{R^{d}V}\int_{B_{R}(x)}|f(y)|dy > \frac{1}{R^dV}\int_{B_{r}(0)}|f(y)|dy
$$
Notice the inequality $\frac{1}{|x| + r}>\frac{1}{|x|(1+r)}$, then take $c = \frac{1}{(1+r)^{d}V}\int_{B_{r}(0)}|f(y)|dy$, we obtain the desired inequality. And $f^{*}$ is not integrable since $\int_{\mathbb{R}^{d}}|f^{*}(x)|dx \geq \int_{|x| \geq 1}|f^{*}(x)|dx = \infty$. \qed 





\end{document}