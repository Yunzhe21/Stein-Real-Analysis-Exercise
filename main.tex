\documentclass[12pt]{article}
\usepackage[margin=1in]{geometry}
\usepackage[all]{xy}


\usepackage{amsmath,amsthm,amssymb,color,latexsym}
\usepackage{geometry}        
\geometry{letterpaper}    
\usepackage{graphicx}

\newtheorem{problem}{Problem}

\newenvironment{solution}[1][\it{Solution}]{\textbf{#1. } }{$\square$}


\begin{document}
\noindent Stein Real Analysis \hfill Chapter 3\\
Yunzhe Zheng. (2025/01)

\hrulefill

\begin{problem}
Suppose $\varphi$ is an integrable function on $\mathbb{R}^{d}$ with $\int_{\mathbb{R}^{d}}\varphi(x)dx = 1$. Set $K_{\delta}(x) = \delta^{-d}\varphi(x/\delta)$, $\delta > 0$. \\

(1). Prove that $\{K_{delta}\}_{\delta>0}$ is a family of good kernels. \\
\indent (2). Assume in addition that $\varphi$ is bounded and supported in a bounded set. Verify that $\{K_{\delta}\}_{\delta>0}$ is an approximation to the identity. \\
\indent (3). Show that Theorem 2.3 holds for good kernels as well.
\end{problem}

\textbf{Proof:} (1). Firstly, $\int_{\mathbb{R}^{d}}K_{\delta}(x)dx = \int_{\mathbb{R}^{d}}\delta^{-d}\varphi(x/\delta)dx$, and by change of variable, it's identical to $\int_{\mathbb{R}^{d}}\varphi(u)du = 1$. Secondly, $\int_{\mathbb{R}^{d}}\left | \delta^{-d} \varphi(x/\delta)\right | = \delta^{-d}\int_{\mathbb{R}^{d}}|\varphi(x/\delta)|dx$, and by change of variable and the fact that $\varphi$ is lebesgue integrable, $\int_{\mathbb{R}^{d}}|\delta^{-d}\varphi(x/\delta)|dx \leq A$ for some $A$ independent of $\delta$. Finally, for fixed $\eta$,  by change of variable, $\int_{\mathbb{R}^{d}}\delta^{-d}|\varphi(x/\delta)|dx = \int_{u>\eta / \delta}|\varphi(u)|du$. Since $\eta / \delta\to\infty$ as $\delta\to 0$, thus the integral approaches 0 by integrability. \\

(2). Suppose that for $|x| > B$, $\varphi(x) = 0$, and $\sup\varphi(x) = M$. Setting $A = MB^{d+1}$, then $\delta^{-d}\varphi(x/\delta)< \frac{M|x/\delta|^{d+1}}{|x|^{d+1}} < \frac{A}{|x|^{d+1}}$ for $\frac{|x|}{\delta}\leq B$, otherwise it's zero. If we pick $B$ large enough ($B > 1$), then $|K_{\delta}(x)| \leq MB^{d+1}\delta^{-d} = A\delta^{-d}$ is obvious. Thus $K_{\delta}(x)$ is an approximation to identity. \\

(3). To prove integrability, we utilize Fubini Theorem, and the convolution is integrable. Secondly, compute

\begin{eqnarray*}
\int_{\mathbb{R}^{d}}(\int_{\mathbb{R}^{d}}f(x-y)K_{\delta}(y)dy - f(x))dx 
&=& \int_{R^{d}}\int_{R^{d}}\left [ f(x-y) - f(x)\right]K_{\delta}(y)dydx \\
&=& \int_{\mathbb{R}^{d}}K_{\delta}(y)\int_{\mathbb{R^{d}}}(f(x-y) - f(x))dxdy \\
&=& \int_{|y| > \eta}K_{\delta}(y)\int_{\mathbb{R}^{d}}(f(x-y) - f(y))dxdy \\
&+& \int_{|y|\leq \eta}K_{\delta}(y)\int_{\mathbb{R}^{d}}(f(x-y)  - f(y))dxdy 
\end{eqnarray*}, where $\eta$ is chosen so that $\int_{\mathbb{R}^{d}}(f(x - y) - f(x))<\epsilon$, thus $\|(f \ast K_{\delta}) - f\|_{1}\to 0$ as $\delta\to 0$. \qed


\begin{problem}
Suppose that $\{K_{\delta}\}$ is a family of kernels that satisfies: \\

(i). $|K_{\delta}(x)|\leq A\delta^{-d}$ for all $\delta > 0$. \\
\indent (ii). $|k_{\delta}(x)|\leq A\delta/|x|^{d+1}|$ for all $\delta > 0$. \\ 
\indent (iii). $\int_{-\infty}^{\infty}K_{\delta}(x)dx = 0$ for all $\delta > 0$. \\

Show that if $f$ is integrable on $\mathbb{R}^{d}$, then $(f\ast K_{\delta})(x) \to 0$ for almost every $x$ as $\delta\to 0$.
\end{problem}

\textbf{Proof:} Notice that $\left | (f \ast K_{\delta})(x) \right | = \left | \int_{\mathbb{R^{d}}} f(x-y)K_{\delta}(y)dy \right | = \left | \int_{\mathbb{R}^{d}}f(x-y)K_{\delta}(y) - \int_{\mathbb{R}^{d}}f(x)K_{\delta}(y)\right |$. Then the proof follows exactly the same as in Theorem 2.1 Chapter 3. \qed

\begin{problem}
    
\end{problem}

\end{document}